%%%%%%%%%%%%%%%%%%%%%%%%%%%%%%%%%%%%%%%%%%%%%%%%%%%%%%%%%%%%%%%%%%%%%%%%%%%%%%%%
\font\eightrm=cmr10 scaled 800
\font\twelverm=cmr10 scaled \magstep1
\font\twelvebf=cmbx10 scaled \magstep1
\font\fourteenrm=cmr10 scaled \magstep2
\font\fourteenbf=cmbx10 scaled \magstep2
\def\chapter#1#2{\bigskip
\hfil{\twelvebf Chapter #1}
\medskip
\hfil{\fourteenbf #2}
\bigskip
\def\thechapter{#1}
\sectioncount=0}
\def\appendix#1#2{\bigskip
\hfil{\twelvebf Appendix #1}
\medskip
\hfil{\fourteenbf #2}
\bigskip
\def\thechapter{#1
\sectioncount=0}}
\def\nochapter#1{\centerline{\fourteenbf #1}\bigskip\sectioncount=0}
\def\thechapter{1}
\countdef\sectioncount=11
\countdef\subsectioncount=13
\countdef\subsubsectioncount=15
\sectioncount=0
\subsectioncount=0
\subsubsectioncount=0
\def\section#1{\advance\sectioncount by 1\subsectioncount=1
\bigskip\noindent{\bf\S\thechapter.\the\sectioncount\ #1}\par
\nobreak\medskip}
\def\subsection#1{\advance\subsectioncount by 1
\bigskip\noindent{\bf#1}\par\nobreak\medskip}
\def\nosection#1{\bigskip\noindent{\bf#1}\par\nobreak\medskip}
\def\sect#1{\advance\sectioncount by1\subsectioncount=0
\bigskip\noindent{\bf\the\sectioncount. #1}\par\medskip}
\def\subsect#1{\advance\subsectioncount by 1\subsubsectioncount=0
\bigskip\noindent{\bf\the\sectioncount.\the\subsectioncount\ #1}\par
\nobreak\medskip}
\def\subsubsect#1{\advance\subsubsectioncount by 1
\bigskip\noindent
{\bf\the\sectioncount.\the\subsectioncount.\the\subsubsectioncount\ #1}\par
\nobreak\medskip}
\def\nosect#1{\bigskip\noindent{\bf#1}\par\nobreak\medskip}
\def\heading#1{\bigskip\noindent{\bf#1}\par\medskip}
\countdef\equationcount=17
\equationcount=0
\def\equation{\global\advance\equationcount by 1
\thechapter.\the\equationcount}
\def\eq{\global\advance\equationcount by 1
\the\equationcount}
\countdef\count=19
\count=0
\def\theorem#1#2{\bigskip\noindent{\bf#1:\ }{\it#2}\par\medskip}
\def\proof{\medskip\noindent{\bf Proof:\ }}
\def\qed{\vtop{\hrule height 10pt width 5pt\bigskip}}
\def\uncatcodespecials{\def\do##1{\catcode`##1=12}\dospecials}
\def\setupverbatim{\tt\def\par{\leavevmode\endgraf}\catcode`\`=\active
\obeylines\uncatcodespecials\obeyspaces}
{\catcode`\`=\active \gdef`{\relax\lq}}
{\obeyspaces\global\let =\ }{\obeylines\global\let^^M=\par}
\def\beginverbatim{\par\begingroup\parindent=0pt\setupverbatim\doverbatim}
{\catcode`|=0 \catcode`\\=12
 |obeylines|gdef|doverbatim^^M#1\endverbatim{#1|endgroup}}
\def\verbatim{\begingroup\setupverbatim\doverb}
\def\doverb#1{\def\next##1#1{##1\endgroup}\next}
\def\start{\ }
\def\bibitem#1#2{\medskip\noindent}
\def\cite#1{[#1]}
\def\date{\the\day\ \ifcase\month\or January\or February\or March\or
April \or May\or June\or July\or August
\or September\or October\or November\or December\fi\ \the\year}
%%%%%%%%%%%%%%%%%%%%%%%%%%%%%%%%%%%%%%%%%%%%%%%%%%%%%%%%%%%%%%%%%%%%%%%%%%%%%%%%
\magnification=\magstep1
\font\bigrm=cmr10 scaled \magstep1
\def\epilog{E{\eightrm PILOG}}
\def\prolog{P{\eightrm ROLOG}}
\def\lisp{C{\eightrm OMMON} L{\eightrm ISP}}
\def\up{$\uparrow$}
%%%%%%%%%%%%%%%%%%%%%%%%%%%%%%%%%%%%%%%%%%%%%%%%%%%%%%%%%%%%%%%%%%%%%%%%%%%%%%%%

\centerline{\bf Converting KIF to SIF}
\centerline{\bf Epistemics Inc.}

\bigskip

Simplified Interchange Format (SIF) is a proper subset of Knowledge Interchange
Format (KIF).  Every expression in SIF is an expression in KIF, but not every
expression in KIF is an expression in SIF.

By and large, SIF is much simpler than KIF.  (Hence the name.)  There are no
definitions and no nonmonotonic rules; there are no explicit quantifiers; there
are no embedded implications; and there are no complicated operators (like {\tt
setofall}, {\tt lambda}, and so forth).  Nested terms are also prohibited, with
the exception of equations and terms that are unique ``designators'' (as
described below). 

These restrictions dramatically simplify the task of automated reasoning.  In
particular, they allow us to implement an inference procedure that is both
efficient and easily understandable.

Despite the subset relationship between SIF and KIF, SIF is every bit as
expressive as KIF, i.e. for any set of KIF sentences, there is an equivalent set
of SIF sentences.  \epilog{} provides subroutines capable of transforming KIF
sentences into equivaent SIF sentences.

The {\tt sif} subroutine converts arbitrary KIF sentences into SIF sentences in
boolean form (i.e. without occurrences of {\tt <=} or {\tt =>}).  It skolemizes,
converts non-primitive nested terms into equational conditions, and then
converts to boolean form.  This is especially usefully for converting negated
goals to SIF.  The following example assumes that {\tt b} is a name
and that {\tt f} and {\tt g} are pseudofunctionals.

\medskip
\beginverbatim
User: (sif '(not (r (f (g ?x)) b)))
Lisp: (OR (NOT (= (G ?X) ?X1)) (NOT (= (F ?X1) ?X2)) (NOT (R ?X2 B)))
\endverbatim
\medskip

The {\tt rules} subroutine converts arbitrary KIF sentences into SIF rules.  It
skolemizes, converts non-primitive nested terms into equational conditions, and
then converts to backward rule form.  In the following examples, we again assume
that {\tt b} is a name and that {\tt f} and {\tt g} are pseudofunctionals.

\medskip
\beginverbatim
User: (car (rules '(r (f (g ?x)) b)))
Lisp: (<= (R ?X2 B) (= (F ?X1) ?X2) (= (G ?X) ?X1))

User: (car (rules '(<= (r (f ?x) ?y) (p ?x) (q (g ?y))))
Lisp: (<= (R ?X1 ?Y) (= (F ?X) ?X1) (P ?X) (= (G ?Y) ?Y1) (Q ?Y1))
\endverbatim
\medskip

The {\tt rules} subroutine is guided by the settings of the variables {\tt
*names*} (default {\tt t}) and {\tt *functionals*} (default {\tt nil}).  If
{\tt *names*} is {\tt t}, all object constants are assumed to be {\it
primitive}.  If it is a list, only those object constants on the list are
assumed to be primitive.  If {\tt *functionals*} is {\tt t}, all function constants
are assumed to produce primitive terms when applied to primitive arguments.  If
it is a list, only those function constants on the list are assumed to produce
primitive terms.  {\tt sif} and {\tt rules} do not modify terms that, according
to the settings of these variables, are determined to be primitive.

\medskip
\beginverbatim
User: (setq *functionals* '(f))
Lisp: (F)

User: (car (rules '(r (f (g ?x)) b)))
Lisp: (<= (R (F ?X1) B) (= (G ?X) ?X1))

User: (setq *functionals* '(g))
Lisp: (G)

User: (car (rules '(r (f (g ?x)) b)))
Lisp: (<= (R ?X1 B) (= (F (G ?X)) ?X1))

User: (setq *functionals* '(f g))
Lisp: (F G)

User: (car (rules '(r (f (g ?x)) b)))
Lisp: (R (F (G ?X)) B)

User: (setq *names* nil)
Lisp: NIL

User: (car (rules '(r (f (g a)) b)))
Lisp: (<= (R (F (G ?X)) ?Y) (= B ?Y) (= A ?X))
\endverbatim
\medskip

\bye

%%%%%%%%%%%%%%%%%%%%%%%%%%%%%%%%%%%%%%%%%%%%%%%%%%%%%%%%%%%%%%%%%%%%%%%%%%%%%%%%
%%%%%%%%%%%%%%%%%%%%%%%%%%%%%%%%%%%%%%%%%%%%%%%%%%%%%%%%%%%%%%%%%%%%%%%%%%%%%%%%
%%%%%%%%%%%%%%%%%%%%%%%%%%%%%%%%%%%%%%%%%%%%%%%%%%%%%%%%%%%%%%%%%%%%%%%%%%%%%%%%
