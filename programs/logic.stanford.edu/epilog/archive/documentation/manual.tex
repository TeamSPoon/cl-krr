%%%%%%%%%%%%%%%%%%%%%%%%%%%%%%%%%%%%%%%%%%%%%%%%%%%%%%%%%%%%%%%%%%%%%%%%%%%%%%%%
\font\eightrm=cmr10 scaled 800
\font\twelverm=cmr10 scaled \magstep1
\font\twelvebf=cmbx10 scaled \magstep1
\font\fourteenrm=cmr10 scaled \magstep2
\font\fourteenbf=cmbx10 scaled \magstep2
\def\chapter#1#2{\bigskip
\hfil{\twelvebf Chapter #1}
\medskip
\hfil{\fourteenbf #2}
\bigskip
\def\thechapter{#1}
\sectioncount=0}
\def\appendix#1#2{\bigskip
\hfil{\twelvebf Appendix #1}
\medskip
\hfil{\fourteenbf #2}
\bigskip
\def\thechapter{#1
\sectioncount=0}}
\def\nochapter#1{\centerline{\fourteenbf #1}\bigskip\sectioncount=0}
\def\thechapter{1}
\countdef\sectioncount=11
\countdef\subsectioncount=13
\countdef\subsubsectioncount=15
\sectioncount=0
\subsectioncount=0
\subsubsectioncount=0
\def\section#1{\advance\sectioncount by 1\subsectioncount=1
\bigskip\noindent{\bf\S\thechapter.\the\sectioncount\ #1}\par
\nobreak\medskip}
\def\subsection#1{\advance\subsectioncount by 1
\bigskip\noindent{\bf#1}\par\nobreak\medskip}
\def\nosection#1{\bigskip\noindent{\bf#1}\par\nobreak\medskip}
\def\sect#1{\advance\sectioncount by1\subsectioncount=0
\bigskip\noindent{\bf\the\sectioncount. #1}\par\medskip}
\def\subsect#1{\advance\subsectioncount by 1\subsubsectioncount=0
\bigskip\noindent{\bf\the\sectioncount.\the\subsectioncount\ #1}\par
\nobreak\medskip}
\def\subsubsect#1{\advance\subsubsectioncount by 1
\bigskip\noindent
{\bf\the\sectioncount.\the\subsectioncount.\the\subsubsectioncount\ #1}\par
\nobreak\medskip}
\def\nosect#1{\bigskip\noindent{\bf#1}\par\nobreak\medskip}
\def\heading#1{\bigskip\noindent{\bf#1}\par\medskip}
\countdef\equationcount=17
\equationcount=0
\def\equation{\global\advance\equationcount by 1
\thechapter.\the\equationcount}
\def\eq{\global\advance\equationcount by 1
\the\equationcount}
\countdef\count=19
\count=0
\def\theorem#1#2{\bigskip\noindent{\bf#1:\ }{\it#2}\par\medskip}
\def\proof{\medskip\noindent{\bf Proof:\ }}
\def\qed{\vtop{\hrule height 10pt width 5pt\bigskip}}
\def\uncatcodespecials{\def\do##1{\catcode`##1=12}\dospecials}
\def\setupverbatim{\tt\def\par{\leavevmode\endgraf}\catcode`\`=\active
\obeylines\uncatcodespecials\obeyspaces}
{\catcode`\`=\active \gdef`{\relax\lq}}
{\obeyspaces\global\let =\ }{\obeylines\global\let^^M=\par}
\def\beginverbatim{\par\begingroup\parindent=0pt\setupverbatim\doverbatim}
{\catcode`|=0 \catcode`\\=12
 |obeylines|gdef|doverbatim^^M#1\endverbatim{#1|endgroup}}
\def\verbatim{\begingroup\setupverbatim\doverb}
\def\doverb#1{\def\next##1#1{##1\endgroup}\next}
\def\start{\ }
\def\bibitem#1#2{\medskip\noindent}
\def\cite#1{[#1]}
\def\date{\the\day\ \ifcase\month\or January\or February\or March\or
April \or May\or June\or July\or August
\or September\or October\or November\or December\fi\ \the\year}
%%%%%%%%%%%%%%%%%%%%%%%%%%%%%%%%%%%%%%%%%%%%%%%%%%%%%%%%%%%%%%%%%%%%%%%%%%%%%%%%
\magnification=\magstep1
\font\bigrm=cmr10 scaled \magstep1
\def\epilog{E{\eightrm PILOG}}
\def\prolog{P{\eightrm ROLOG}}
\def\lisp{C{\eightrm OMMON} L{\eightrm ISP}}
\def\up{$\uparrow$}
\def\label#1{}
%%%%%%%%%%%%%%%%%%%%%%%%%%%%%%%%%%%%%%%%%%%%%%%%%%%%%%%%%%%%%%%%%%%%%%%%%%%%%%%%

\start\hfill\date
\vfill
\centerline{{\fourteenbf EPILOG 2.0 for LISP}}\par
\vfill
\centerline{{\fourteenbf Draft}}\par
\vfill
\noindent (c) Copyright 1993-1995 Epistemics.  Permission is
hereby granted to anyone to make copies for any non-commercial use.
\eject
%%%%%%%%%%%%%%%%%%%%%%%%%%%%%%%%%%%%%%%%%%%%%%%%%%%%%%%%%%%%%%%%%%%%%%%%%%%%%%%%

\centerline{\twelvebf Table of Contents}

\def\chap#1#2{\bigskip
\noindent{\bf #1}\leaders\hbox to 1em{\hss.\hss}\hfill#2\par}
\def\sect#1#2{#1\leaders\hbox to 1em{\hss.\hss}\hfill#2\par}

\chap{1. Introduction}{n}
\sect{Overview}{n}
\sect{Loading \epilog{}}{n}
\sect{Reader's Guide}{n}

\chap{2. Simplified Interchange Format}{n}
\sect{Overview}{n}
\sect{Atoms}{n}
\sect{Terms}{n}
\sect{Sentences}{n}

\chap{3. Categorizing and Manipulating Expressions}{n}

\chap{4. Pattern Matching}{n}

\chap{5. Theories}{n}
\sect{Overview}{n}
\sect{Indexing Subroutines}{n}
\sect{Theory Manipulation Subroutines}{n}
\sect{Composite Theories}{n}
\sect{Theory Composition Subroutines}{n}
\sect{Manipulation Subroutines for Composite Theories}{n}
\sect{Viewing Theories}{n}
\sect{Secondary Storage}{n}

\chap{6. Inference}{n}
\sect{Backward Chaining}{n}
\sect{Forward Chaining}{n}
\sect{Search Control}{n}
\sect{Reasoning with Equality}{n}
\sect{Procedural Attachments}{n}
\sect{Tracing}{n}

\chap{7. Miscellaneous}{n}

\chap{Bibliography}{n}

\chap{EPILOG Variables}{n}

\chap{EPILOG Subroutines}{n}

\chap{Index}{n}

\vfill\eject
%%%%%%%%%%%%%%%%%%%%%%%%%%%%%%%%%%%%%%%%%%%%%%%%%%%%%%%%%%%%%%%%%%%%%%%%%%%%%%%%

\chapter{1}{Introduction}

\section{Overview}

\epilog{} is a library of \lisp{} subroutines that implement an efficient
inference procedure for information encoded in SIF (Simplified Interchange
Format).

The inference procedure used in \epilog{} is based on a reasoning technique
called {\it model elimination}.  The procedure closely resembles that of
\prolog{}; but, unlike that that of \prolog{}, the procedure used in \epilog{}
is sound and complete for the entire language, i.e. all consequences the
procedure derives are correct and the procedure can derive all correct
consequences of the information it is given.

SIF, the language supported by \epilog{}, is a subset of KIF (Knowledge
Interchange Format), i.e. all expressions in SIF are expressions in KIF, but the
reverse is not true.  Despite this relationship, SIF is every bit as expressive
as KIF, i.e. for any set of KIF sentences, there is an equivalent set of SIF
sentences.  Furthermore, this set of sentences can be derived automatically, and
\epilog{} includes subroutines capable of performing this transformation.

Given the inference procedure defined in \epilog{} and these transformation
subroutines, it is possible to build a sound and complete inference procedure
for all of KIF.  \epilog{} also includes a sound and complete information
procedure for KIF implemented in this way.

\section{Loading EPILOG}

The code for \epilog{} is contained in a single file in the {\tt code}
subdirectory on the \epilog{} disk.  This subdirectory contains different files
for different implementations of \lisp{}.  The filename extension designates
the implementation for which the file is appropriate.  For example, the
filename {\tt epilog.mcl} is the M{\eightrm ACINTOSH} \lisp{} version.  Although
the object code is different for each implementation of \lisp{}, the
functionality of the library is the same in all cases.

As a subroutine library, \epilog{} does not run in standalone fashion.  It is
designed to be loaded into a running version of \lisp{} using \lisp{}'s
{\tt load} routine.  Once the library is loaded, all of the variables and
subroutines are available for use; no further set up is required.

Note, however, that \epilog is an extension of the \epilog{} subroutine library
and, as such, requires that that library be loaded in order to function
properly.  Although in some implementations it is possible to load these two
libraries in either order, it is recommended that the \epilog{} library be loaded
first.

\section{Reader's Guide}

This manual provides full details on the \epilog{} library.  Chapter~2
defines SIF.  Chapter~3 describes the \epilog{} facilities for transforming KIF
sentences into SIF sentences.  Chapter~4 describes the SIF inference subroutines
provided by \epilog{}.  There are two appendices.  The first gives an
alphabetical listing of all variables in \epilog{} together with brief
descriptions.  The first gives an alphabetical listing of all subroutiness in
\epilog{}, also with brief descriptions.

\vfill\eject
%%%%%%%%%%%%%%%%%%%%%%%%%%%%%%%%%%%%%%%%%%%%%%%%%%%%%%%%%%%%%%%%%%%%%%%%%%%%%%%%

\chapter{2}{Simplified Interchange Format}

\section{Overview}

SIF is a prefix version of the language of first order predicate calculus with
various extensions to enhance its expressiveness.  In \epilog{}, SIF expressions
are represented as \lisp{} atoms and lists (but not dotted pairs).

First and foremost, SIF provides for the expression of simple data.  For example,
the sentences shown below encode three tuples in a personnel database.  The first
element in each list is the name of the relation (in this case {\tt salary}).
Next comes the social security number of each individual, then the department
within which the individual works, and, finally, the individual's salary.

\medskip
{\tt (salary 015-46-3946 widgets 72000)}\par
{\tt (salary 026-40-9152 grommets 36000)}\par
{\tt (salary 415-32-4707 fidgets 42000)}\par
\medskip

SIF includes a variety of logical operators to assist in the encoding of logical
information (such as negations, disjunctions, rules, and so forth).  The
expression shown below is an example of a logical sentence in SIF.  It defines
a grandparent as a parent of a parent.

\medskip
{\tt (<= (grandparent ?x ?z) (parent ?x ?y) (parent ?y ?z))}\par
\medskip

One of the distinctive features of SIF is its ability to encode knowledge about
knowledge, using the \up{} and {\tt ,} operators and related vocabulary.  For
example, the following sentence asserts that Joe is interested in receiving
triples in the salary relation.  The use of commas signals that the variables
should not be taken literally.  Without the commas, this sentence would say that
Joe is interested in the sentence {\tt (salary ?x ?y ?z)} literally instead of
its instances.

\medskip
{\tt (interested joe \up(salary ,?x ,?y ,?z))}\par
\medskip

In this chapter, we examine SIF in detail. We first look at various
categorizations of atoms in SIF.  Given these categorizations, we then define
the terms of the language.  Finally, we define the sentences of the
language.  (Terms are used to denote objects in the world being described;
sentences and forms are used to express facts about the world.)

\section{Atoms}

In SIF, all \lisp{} atoms are assigned to exactly one of three catagories:
variables, operators, and constants.

There are two types of variables.  {\it Individual variables} are symbols that
begin with {\tt ?}.  {\it Sequence variables} are symbols that begin with {\tt
@}.  Individual variables are used in quantifying over individual objects. 
Sequence variables are used in quantifying over sequences of objects.

{\it Operators} are used in forming complex expressions of various sorts.  There
are three types of operators -- term operators (the symbols in the first column
below), sentences operators (the symbols in the second column), and KIF
operators (the symbols in the third and fourth columns).

\medskip
\centerline{\vbox{\halign{\strut{\tt #}\hfil\cr
listof\cr
quote\cr
\cr
\cr
\cr
\cr
\cr
\cr
\cr
\cr}}\hfil
\vbox{\halign{\strut{\tt #}\hfil\cr
=\cr
/=\cr
not\cr
and\cr
or\cr
=>\cr
<=\cr
\cr
\cr
\cr}}\hfil
\vbox{\halign{\strut{\tt #}\hfil\cr
setof\cr
if\cr
cond\cr
the\cr
setof\cr
setofall\cr
kappa\cr
lambda\cr
forall\cr
exists\cr}}\hfil
\vbox{\halign{\strut{\tt #}\hfil\cr
defobject\cr
deffunction\cr
defrelation\cr
<<=\cr
=>>\cr
\cr
\cr
\cr
\cr
\cr}}}
\medskip

Term operators and sentence operators are used in forming complex SIF
epxressions.  Term operators are used in forming complex terms.  Sentence
operators are used in forming complex sentences.  Operators of the third type,
KIF operators, are not used in SIF.  The reason is a subtle.  The inference
subroutines in \epilog{} do not give these operators any special treatment, and
so they {\it can} in practice be used as constants in expressions without
untoward consequences.  But then we would have expressions that are illegal in
KIF and that might not be acceptable to other programs.  By legislating them out
of the language, we guarantee that any legal expression in SIF is a legal
expression in KIF.

By convention, constants are divided into four categories -- {\it object
constants}, {\it function constants}, {\it relation constants}, and logical
constants.  Object constants denote objects in the world being described. 
Function constants denote mappings from those objects into those objects. 
Relation constants denote relations on those objects.  Logical constants
are similar to relation constants but have no ``arguuments''.  The types of
constants need not be declared before they are used; those types are determined
implicitly by the places in which they are used.

SIF further subdivides objects constants into two types -- {\it names} and {\it
pseudonyms}.  Intuitively, a name is a unique designator for an object in the
world being described.  An object can have at most one name; and so, if two
names are distinct, they must by definition denote distinct objects.  By
contrast, an object can have any number of pseudonyms.

In \epilog{}, all characters, strings, and numbers are names.  The status of
other constants is determined by the value of the variable {\tt *names*}.  If
the value of {\tt *names*} is a list, the elements of the list are considered
names as well.  If the value is a non-list, all object constants are treated as
names.  The initial value is {\tt t}, i.e. all object constants are
considered names.

SIF also subdivides function constants into two types -- {\it functionals} and
{\it pseudofunctionals}.  Intuitively, a functional is used to form complex
designators for objects, as described a few paragraphs below.  In analogy with
names and pseudonyms, the categorization of constants into functionals and
pseudofunctionals is determined by the value of a variable, in this case {\tt
*functionals*}.  The initial value in this case is {\tt nil}, i.e. all function
constants are considered to be pseudofunctionals.

\section{Terms}

There are two types of terms -- {\it designators} and {\it descriptors}.  The
distinction here is {\it roughly} equivalent to the distinction between names
and pseudonyms.  There are five types of designators -- individual variables,
names, quotations, list terms, and functional designators.  There are two types
of descriptors -- pseudonyms and functional descriptors.

{\it Quotations} involve the {\tt quote} operator and an arbitrary list
expression.  The embedded expression can be an arbitrary list structure; it need
{\it not} be a legal expression in SIF.  Remember that the \lisp{} reader
converts strings of the form {\tt '$\epsilon$} into {\tt (quote $\epsilon$)}. 
The following are legal quotations.

\medskip
{\tt (quote a)}\par
{\tt (quote (p (f a) b))}\par
{\tt (quote (p ?x b))}\par
{\tt (quote (p (quote a) b))}\par
{\tt (quote (<= (>= ?x ?y) (> ?x ?y)))}\par
\medskip

A {\it list term} consists of the {\tt listof} operator and a finite list of
terms, terminated by an optional sequence variable.  Note that the \epilog{}
subroutine library used by \epilog{} redefines the \lisp{} readtable so that
expressions starting with the uparrow character (\up) are tranformed
into quoted symbols and lists, in a manner analogous to that of expressions
involving backquote ({\tt `}).  See the \epilog{} manual for more details.  The
following are list terms.

\medskip
{\tt (listof a b c)}\par
{\tt (listof a (quote b) c)}\par
{\tt (listof (quote a) ?x c)}\par
{\tt (listof a (listof ?x ?y ?z) c)}\par
\medskip

A {\it functional designator} consists of a functional and an arbitrary number
of designators, terminated by an optional sequence variable.  Note that there is
no syntactic  restriction on the number of argument terms -- the same function
constant can be applied to any number of arguments.  If {\tt a} and {\tt b} are
names and {\tt f} is a functional, then the following are functional
designators.

\medskip
{\tt (f a b)}\par
{\tt (f ?x b)}\par
{\tt (f (f ?x b) b)}\par
\medskip

A {\it functional descriptor} consists of a pseudofunctional and an arbitrary
number of terms of any sort, terminated by an optional sequence variable. 
Again there is no syntactic  restriction on the number of argument terms -- the
same function constant can be applied to any number of arguments.  If {\tt a}
and {\tt b} are names, {\tt f} is a functional, and {\tt g} is a
pseudofunctional, then the following are functional designators.

\medskip
{\tt (g a b)}\par
{\tt (g ?x b)}\par
{\tt (g (f ?x b) b)}\par
\medskip

Note that, while functional designators may be nested within other expressions,
functional descriptors may not.  For example, if {\tt g} is a pseudofunctional,
the expression {\tt (g (g ?x))} is illegal.  Fortunately, such expressions {\it
can} be written in ``expanded'' form, as described in the next chapter.

\section{Sentences}

From terms, we can build sentences.  There are nine types of sentences --
logical constants, equations, inequalities, relational sentences, negations,
conjunctions, disjunctions, forward rules, and backward rules.

{\it Logical constants} are simple constants.  They are used without arguments
or other syntax to denote conditions in the world.  The following examples are
suggestive.

\medskip
{\tt london-is-rainy}\par
{\tt san-francisco-is-foggy}\par
{\tt los-angeles-is-sunny}\par
\medskip

An {\it equation} is an expression of the form {\tt (= $\tau_1$ $\tau_2$)},
where $\tau_1$ is an arbitrary term and $\tau_2$ is a designator.  If {\tt a}
and {\tt b} are names, {\tt f} is a functional, and {\tt g} is a
pseudofunctional, then the following are legal equations.

\medskip
{\tt (= (g a) b)}\par
{\tt (= (g a) (f b))}\par
{\tt (= (g ?x) ?x)}\par
{\tt (= (g (f ?x)) ?x)}\par
\medskip

An {\it inequality} is an expression of the form {\tt (/= $\tau_1$ $\tau_2$)},
where $\tau_1$ is an arbitrary term and $\tau_2$ is a designator.   If {\tt a}
and {\tt b} are names, {\tt f} is a functional, and {\tt g} is a
pseudofunctional, then the following are legal inequalities.

\medskip
{\tt (/= (g a) b)}\par
{\tt (/= (g a) (f b))}\par
{\tt (/= (g ?x) ?x)}\par
{\tt (/= (g (f ?x)) ?x)}\par
\medskip

A {\it relational sentence} consists of a relation constant and an arbitrary
number of designators, terminated by an optional sequence variable.  As with
functional designators, there is no syntactic restriction on the number of
argument terms in a relation sentence -- the same relation constant can be
applied to any finite number of arguments.  If {\tt a} and {\tt b} are
names, {\tt f} is a functional, and {\tt r} is a relation constant, then the
following are relational sentences.

\medskip
{\tt (r a b)}\par
{\tt (r ?x b)}\par
{\tt (r (f ?x b) b)}\par
\medskip

A {\it negation} is a sentence of the form {\tt (not $\sigma$)}, where $\sigma$
is a relational sentence.   Note that negations cannot be nested.  If {\tt a}
and {\tt b} are names, {\tt f} is a functional, and {\tt r} is a relation
constant, then the following are negations.

\medskip
{\tt (not (r a b))}\par
{\tt (not (r ?x b))}\par
{\tt (not (r (f ?x b) b))}\par
\medskip

A {\it conjunction} is a sentence of the form {\tt (and $\sigma_1$ ...
$\sigma_n$)}, where the $\sigma_i$ are sentences of any sort except forward
rules or backward rules.  If {\tt a}, {\tt b}, and {\tt c} are names, {\tt f} is
a functional, and {\tt r} is a relation constant, then the following are
conjunctions.

\medskip
{\tt (and (r a b) (r b c))}\par
{\tt (and (and (r a b) (r b c)) (r a c)}\par
\medskip

A {\it disjunction} is a sentence of the form {\tt (and $\sigma_1$ ...
$\sigma_n$)}, where the $\sigma_i$ are sentences of any sort except forward
rules or backward rules.  If {\tt a}, {\tt b}, and {\tt c} are names, {\tt f} is
a functional, and {\tt r} is a relation constant, then the following are
disjunctions.

\medskip
{\tt (or (r a b) (r b c))}\par
{\tt (or (and (r a b) (r b c)) (r a c)}\par
{\tt (or (or (r a b) (r b c)) (r a c)}\par
\medskip

A {\it forward rule} is a sentence of the form {\tt (=> $\sigma$ $\sigma_1$ ...
$\sigma_n$ $\sigma'$)}, where $\sigma$ is an equation, inequality, relational
sentence, or negation, where $\sigma'$ is an equation, inequality, relationsl
sentence, or negation, and where the $\sigma_i$ are sentences of any sort except
forward rules or backward rules.

\medskip
{\tt (=> (m ?x) (p ?x))}\par
{\tt (=> (m ?x) (p ?x) (= (h ?x) 2))}\par
\medskip

A {\it backward rule} is a sentence of the form {\tt (<= $\sigma$ $\sigma_1$ ...
$\sigma_n$)}, where $\sigma$ is an equation, inequality, relational sentence,
or negation, where the $\sigma_i$ are sentences of any sort except forward rules
or backward rules.

\medskip
{\tt (<= (p ?x @l) (q ?x) (p @l))}\par
{\tt (<= (r ?x) (p ?x) (q ?x))}\par
{\tt (<= (r ?x) (and (p ?x) (q ?x)))}\par
{\tt (<= (not (r ?x)) (or (not (p ?x)) (not (q ?x))))}\par
{\tt (<= (= (g ?x) a) (r ?x))}\par
\medskip

\vfill\eject
%%%%%%%%%%%%%%%%%%%%%%%%%%%%%%%%%%%%%%%%%%%%%%%%%%%%%%%%%%%%%%%%%%%%%%%%%%%%%%%%

\chapter{3}{Pattern Matching}

This section describes the different matching subroutines provided in \epilog{}. In
addition it describes the subroutines for: checking if an expression is a
variable, standardizing the variables in an expression, and instantiating an
expression for some variable bindings. It is important to understand the different
matching subroutines since the behavior of many \epilog{} subroutines can be
parameterized by selecting different matching subroutines.

Given two expressions, a matching operation identifies if the two expressions
are similar. The matching subroutines are ordered hierarchically, from the
very specific to the most general. Below is the set of matching subroutines
provided in \epilog{}, listed from the most specific to the most general:

\medskip
\beginverbatim
eql
equal
identp
samep
samelist
matchp
matcher
instp
instantiator
mgup
mgu
unifyp
\endverbatim
\medskip

The semantics of the first two matching operations should be clear to those
familiar with Lisp.  The operation {\tt eql} tests whether the two expressions
are the same object, while the operation {\tt equal} tests whether the two
expressions have the identical structure and have the same objects at their
leaves.

\medskip
\beginverbatim
User: (setq a '(one two))
Lisp: (ONE TWO)

User: (setq b '(one two))
Lisp: (ONE TWO)

User: (eql a a)
Lisp: T

User: (equal a a)
Lisp: T

User: (eql a b)
Lisp: NIL

User: (equal a b)
Lisp: T
\endverbatim
\medskip

The subroutine {\tt identp} is used to check whether two quoted expressions
denote the same object.  Recall from the previous chapter that the KIF
language restricts the interpretation of quoted expressions to be the
expression that is quoted. For example, the term {\tt (quote (=> (apple ?x)
(fruit ?x)))} denotes the sentence {\tt (=> (apple ?x) (fruit ?x))}. The
function constant {\tt quote} is opaque, and treats its arguments literally. So
{\tt identp} is similar to {\tt equal}, however, there is one generalization-- a
quoted list with elements $\alpha_{i}$ matches a sequence of quoted expressions
$\alpha_{i}$. The following examples illustrate this.

\medskip
\beginverbatim
User: (identp '?x '?x)
Lisp: T

User: (identp '@y '@y)
Lisp: T

User: (identp '(quote ?x) '(quote ?y))
Lisp: NIL

User: (identp nil '(quote nil))
Lisp: NIL

User: (identp '(p '(g a) b) '(p (listof 'g 'a) b))
Lisp: T

User: (identp '(p ?x b) '(p a b))
Lisp: NIL
\endverbatim
\medskip

The routine {\tt identp} returns {\tt T} if the two expressions are meta-identical,
and returns {\tt NIL} otherwise. Note that the quoted expression {\tt '(brother
sam joe)} is meta-identical to the expression {\tt (listof 'brother 'sam 'joe)}. The
second expression is a sequence of the elements of the first expression quoted.
The remaining matching routines {\tt samep, samelist, matchp, matcher, instp,
mgup, mgu} and {\tt unifyp} are generalizations of the {\tt identp} matching
subroutine; i.e., they support the matching operations illustrated above.

\label{samep}
The matching subroutine {\tt samep} is used to check whether two expressions are
the same up to variable renaming.  That is, the structures of the two
expressions are identical, and an atom at a leaf in one expression is identical
to the atom at the corresponding leaf in the second expression, and the
variable at a leaf in one expression is mapped to a unique variable at the
corresponding leaf in the second expression.  The mapping between the
variables of the two expressions must be one-to-one. The variables of the two
expressions are considered disjoint.

\medskip
\beginverbatim
User: (samep '(p ?x b) '(p ?y b))
Lisp: T

User: (samep '(p ?x b) '(p a b))
Lisp: NIL

User: (samep '(p ?x ?y) '(p ?y ?x))
Lisp: T

User: (samep '(p ?x ?y) '(p ?w ?w))
Lisp: NIL

User: (samep '(p ?x @y) '(p ?x ?y))
Lisp: NIL

User: (samep '(p ?x @y) '(p ?y @x))
Lisp: T
\endverbatim
\medskip

\label{samelist}
The matching subroutine {\tt samelist} is an extension of {\tt samep}, which returns
a binding list if the two expressions are the same. It returns {\tt NIL} if the
two expressions are not the same, and otherwise it returns a binding list. A
{\tt binding list} defines the substitutions for some variables. A binding list
is a Lisp {\tt alist}, where the first element of each pair is the variable whose
substitution is being defined, and the second element of the pair is its binding.
The default binding list with no substitutions is {\tt ((T . T))}.

\medskip
\beginverbatim
User: (samelist '(q e d) '(q e d))
Lisp: ((T . T))

User: (samelist '(p ?x b) '(p ?y b))
Lisp: ((?X . ?Y) (T . T))

User: (samelist '(p ?x b) '(p a b))
Lisp: NIL

User: (samelist '(p ?x ?y) '(p ?y ?x))
Lisp: ((?Y . ?X) (?X . ?Y) (T . T))

User: (samelist '(p ?x ?y) '(p ?a ?b))
Lisp: ((?Y . ?B) (?X . ?A) (T . T))

User: (samelist '(p ?x ?y) '(p ?w ?w))
Lisp: NIL

User: (samelist '(p ?x @y) '(p ?x ?y))
Lisp: NIL

User: (samelist'(p ?x @y) '(p ?y @x))
Lisp: ((@Y . @X) (?X . ?Y) (T . T))
\endverbatim
\medskip

The routine returns {\tt NIL} if the two expressions are not the same, and
otherwise it returns a binding list for the variables in the first expression.
If the variables in the first expression are instantiated with this binding list,
then the two expressions are meta-equivalent.

\label{matchp}
The routine {\tt matchp} checks if the second expression is an instance of the
first expression, ignoring one-to-one variable renaming. The two expressions can
be matched if there is a binding list for the first expression, such that the
instantiation of the first expression by this binding list is the {\tt same} as
the second expression. The subroutine returns {\tt NIL} if the two expressions do
not match, and otherwise it returns {\tt T}. The following examples illustrate
the operation of the {\tt matchp} routine.

\medskip
\beginverbatim
User: (matchp '(p ?x b) '(p a b))
Lisp: T

User: (matchp '(p ?x ?y) '(p ?y x))
Lisp: T

User: (matchp '(parents bill @x) '(parents bill ralph mary))
Lisp: T

User: (matchp '(p a b) '(p ?x b))
Lisp: NIL

User: (matchp '(p a b) '(p c ?x))
Lisp: NIL

User: (matchp '(p ?x b) '(p a ?x))
Lisp: NIL

User: (matchp '(p ?x b) '(p ?x b))
Lisp: T
\endverbatim
\medskip

\label{matcher}
The matching subroutine {\tt matcher} is an extension of {\tt matchp}, which
returns a binding list if the two expression match. The two expressions can be
matched if there is a binding list for the first expression, such that the
instantiation of the first expression by this binding list is the {\tt same} as
the second expression. The subroutine returns {\tt NIL} if the two expressions do
not match, and otherwise it returns the binding list for the first expression.
The following examples illustrate the operation of the {\tt matcher} routine.

\medskip
\beginverbatim
User: (matcher '(p ?x b) '(p a b))
Lisp: ((?X . A) (T . T))

User: (matcher '(p ?x ?y) '(p ?y x))
Lisp: ((?Y . X) (?X . ?Y) (T . T))

User: (matcher '(parents bill @x) '(parents bill ralph mary))
Lisp: ((@X RALPH MARY) (T . T))

User: (matcher '(p a b) '(p ?x b))
Lisp: NIL

User: (matcher '(p a b) '(p c ?x))
Lisp: NIL

User: (matcher '(p ?x b) '(p a ?x))
Lisp: NIL

User: (matcher '(p ?x b) '(p ?x b))
Lisp: ((?X . ?X) (T . T))
\endverbatim
\medskip

\label{instp}
The routine {\tt instp} is identical to {\tt matchp}, except that it checks if
the first expression is an instance of the second expression, i.e., {\tt (instp
<a> <b>)} is identical to {\tt (matchp <b> <a>)}. The routine returns {\tt T} if
the first expression is an instance of the second expression (ignoring one-to-one
variable renaming), and otherwise it returns {\tt NIL}. The following examples
illustrate the {\tt instp} matching operation.

\medskip
\beginverbatim
User: (instp '(p a b) '(p ?x b))
Lisp: T

User: (instp '(p ?x b) '(p a b))
Lisp: NIL

User: (instp '(p a b c)  '(p a @x))
Lisp: T
\endverbatim
\medskip

\label{instantiator}
The matching subroutine {\tt instantiator} is an extension of {\tt instp}, which
returns a binding list if the two expressions match. The following examples
illustrate the operation of the {\tt instantiator} routine.

\medskip
\beginverbatim
User: (instantiator '(p a b) '(p ?x b))
Lisp: ((?X . A) (T . T))

User: (instantiator '(p ?x b) '(p a b))
Lisp: NIL

User: (instantiator '(p a b c)  '(p a @x))
Lisp: ((@x B C) (T . T))
\endverbatim
\medskip

\label{mgup}
The routine {\tt mgup} checks if there is a most general unifier of its two
arguments. Identical variables in the two expressions are treated as the same
object; e.g., if {\tt ?x} occurs in both expressions, then it is the same
variable in both expressions. It is possible to unify the two expressions if
there exists a binding list, such that the instantiation of the first expression
by the binding list is meta-equivalent to the instantiation of the second
expression by the binding list. The routine returns {\tt NIL} if the two
expressions cannot be unified, and otherwise it returns {\tt T}. The following
examples illustrate the operation of the {\tt mgup} routine.

\medskip
\beginverbatim
User: (mgup '(p ?x b) '(p a ?y))
Lisp: T

User: (mgup '(p ?x b) '(p a ?x))
Lisp: NIL

User: (mgup '(p ?x b) '(p ?x b))
Lisp: T

User: (mgup '(p ?x (f ?y)) '(p (g ?u) ?v))
Lisp: T

User: (mgup '(+ (2 ?x) @y) '(+ (?x ?z) 4 5))
Lisp: T

User: (mgup '(+ (2 ?x) @y) '(+ (?x 3) 4 5))
Lisp: NIL
\endverbatim
\medskip

\label{mgu}
The matching subroutine {\tt mgu} is an extension of {\tt mgup} which returns a
binding list if its arguments unify. It is possible to unify the two expressions
if there exists a binding list, such that the instantiation of the first
expression by the binding list is {\tt equal} to the instantiation of the second
expression by the binding list. The routine returns {\tt NIL} if the two
expressions cannot be unified, and otherwise it returns the binding list that
unifies the two expressions. The following examples illustrate the operation of
the {\tt mgu} routine.

\medskip
\beginverbatim
User: (mgu '(p ?x b) '(p a ?y))
Lisp: ((?Y . B) (?X . A) (T . T))

User: (mgu '(p ?x b) '(p a ?x))
Lisp: NIL

User: (mgu '(p ?x b) '(p ?x b))
Lisp: ((T . T))

User: (mgu '(p ?x (f ?y)) '(p (g ?u) ?v))
Lisp: ((?V F ?Y) (?X G ?U) (T . T))

User: (mgu '(+ (2 ?x) @y) '(+ (?x ?z) 4 5))
Lisp: ((@Y 4 5) (?Z . 2) (?X . 2) (T . T))

User: (mgu '(+ (2 ?x) @y) '(+ (?x 3) 4 5))
Lisp: NIL
\endverbatim
\medskip

Note that {\tt mgu} returns the most general unifier of two expressions. 
For example, the expressions {\tt (p ?x b)} and {\tt (p ?x b)} can be unified
with the null binding list-- the most general
unifier. However, the two expressions can also be unified by the binding list
{\tt ((?x .  $expression$) (T . T))}, where $expression$ is an arbitrary
expression.  Although the two expressions will still be meta-equivalent with
this instantiation, the resulting expression {\tt (p $expression$ b)} is not as
general as the original expression {\tt (p ?x b)}.

The following example illustrates a potential problem in the unification
process.

\medskip
\beginverbatim
User: (let ((*occurcheck* nil)) (mgu '(p ?x (f ?x)) '(p (g ?y) ?y)))
Lisp: ((?Y F ?X) (?X G ?Y) (T . T))
\endverbatim
\medskip

At first you might think that this answer is correct (this is what \epilog{} returns!).
This is in fact what most \prolog{} systems do.  However, this answer is
incorrect since the previous binding list is circular.  The variable {\tt ?x}
is bound to the expression {\tt (g ?y)}, where the variable {\tt ?y} is
itself bound to the expression {\tt (f ?x)}.  Thus, the variable {\tt ?x} is
bound to the expression {\tt (g (f ?x))}.  This is not a valid binding list
since it is recursive-- a variable is bound to an expression containing
itself.  The previous two expressions cannot be unified.

\label{*occurcheck*}
This is called the {\it occurs check} problem, and a more efficient algorithm
can be used if one is willing to live with the wrong answer with recursive
variable bindings (most \prolog{} systems make this tradeoff). In \epilog{} it is
possible for the user to select if the occurs check is to be performed or not.
If the Lisp variable {\tt *occurcheck*} is bound to a non-nil value, then the
occurs check is performed; otherwise, it is not. The initial value of the variable
{\tt *occurcheck*} is {\tt T}. The following example illustrates this.

\medskip
\beginverbatim
User: (mgu '(p ?x (f ?x)) '(p (g ?y) ?y))
Lisp: NIL
\endverbatim
\medskip

\label{unifyp}
The routine {\tt unifyp} is similar to {\tt mgup}, except that the variables
in the two expressions are considered disjoint.  Instead of renaming the
variables in one of the expressions, the routine separates the variables in
the two expressions using additional internal data structures preventing
the potentially expensive need for copying expressions. The following examples
illustrate the operation of the {\tt unifyp} matching routine.

\medskip
\beginverbatim
User: (unifyp '(red ?x) '(red @x))
Lisp: T

User: (unifyp '(p ?x) '(?x q))
Lisp: T

User: (unifyp '(p (f x) (q y)) '(p ?x (q ?x)))
Lisp: NIL

User: (unifyp '(+ ?x 3 (* 2 y)) '(+ 2 3 (* 2 ?x)))
Lisp: T
\endverbatim
\medskip

\label{varp}
The subroutine {\tt varp} takes one argument and returns {\tt T} if the argument
is a variable, and returns {\tt NIL} otherwise. Examples:

\medskip
\beginverbatim
User: (varp 'red)
Lisp: NIL

User: (varp '(p q))
Lisp: NIL

User: (varp '?x)
Lisp: T

User: (varp '@x)
Lisp: T
\endverbatim
\medskip

\label{indvarp} \label{seqvarp}
The subroutine {\tt indvarp} checks if its argument is a variable starting with
the character {\tt ?}. It returns {\tt T} if its argument is a variable
starting with {\tt ?}, and otherwise it returns {\tt NIL}. Similarly, the
subroutine {\tt seqvarp} checks if its argument is a sequence variable starting
with the character {\tt @}. It returns {\tt T} if its argument is a sequence
variable, and otherwise it returns {\tt NIL}.

\label{stdize}
The subroutine {\tt stdize} is used to standardize the variables in an expression.
Standardization involves replacing every variable with a new variable of the
same type. Individual variables are replaced with new individual variables, and
sequence variables are replaced with new sequence variables.

\medskip
\beginverbatim
User: (stdize '(forall (?x) (=> (apple ?x) (fruit ?x))))
Lisp: (FORALL (?1) (=> (APPLE ?1) (FRUIT ?1)))

User: (stdize '(red ?x @y))
Lisp: (RED ?2 @1)
\endverbatim
\medskip

\label{plug}
The subroutine {\tt plug} takes an expression and a binding list as arguments. It
returns a copy of the expression with its variables instantiated by the binding
list. Note that the original expression is returned without copying if the binding
list is empty. The following example illustrates the {\tt plug} routine:

\medskip
\beginverbatim
User: (mgu '(father ?x ?y) '(father fred sam))
Lisp: ((?Y . SAM) (?X . FRED) (T . T))

User: (plug '(parent ?x ?y) '((?Y . SAM) (?X . FRED) (T . T)))
Lisp: (PARENT FRED SAM)

User: (plug '(parent ?u jackson) '((?Y . SAM) (?X . FRED) (T . T)))
Lisp: (PARENT ?Z JACKSON)
\endverbatim
\medskip

\label{plugstdize}
The subroutine {\tt plugstdize} takes an expression and a binding list as arguments.
It returns a copy of the expression with its variables instantiated by the binding
list.  The variables in the expression that are not bound in the binding list are
replaced with new names, as is done by {\tt stdize}. The following examples
illustrate the {\tt plugstdize} routine:

\medskip
\beginverbatim
User: (mgu '(father ?x ?y) '(father fred sam))
Lisp: ((?Y . SAM) (?X . FRED) (T . T))

User: (plugstdize '(parent ?x ?y) '((?Y . SAM) (?X . FRED) (T . T)))
Lisp: (PARENT FRED SAM)

User: (plugstdize '(parent ?u jackson) '((?Y . SAM) (?X . FRED) (T . T)))
Lisp: (PARENT ?U JACKSON)
\endverbatim
\medskip

\vfill\eject
%%%%%%%%%%%%%%%%%%%%%%%%%%%%%%%%%%%%%%%%%%%%%%%%%%%%%%%%%%%%%%%%%%%%%%%%%%%%%%%%

\chapter{4}{Theories}

\section{Overview}

\epilog{} provides capabilities for creating, modifying, and destroying
``theories'', i.e. sets of sentences encoded in KIF.  The library includes two
layers of subroutines -- indexing subroutines and theory manipulation
subroutines.

The indexing subroutines provide support for theory access and modification. 
With these subroutines, it is possible to add sentences to a theory, delete
sentences from a theory, and find all sentences containing a given atom.

The theory manipulation subroutines provide similar capabilities, but they are
based on patterns rather than atoms.  For example, the {\tt save} routine checks
whether a matching sentence already exists in a theory before adding it to the
theory; the {\tt truep} routine determines whether the theory contains a
sentence that matches a given pattern; the {\tt drop} removes all sentences
matching a given pattern.

Theories are usually accessed and modified using the theory manipulation
subroutines.  However, in certain circumstances, it is sensible to use the
indexing routines instead.  For example, a user might want to call the indexing
subroutines directly and thereby avoid the overhead of pattern matching; or the
user might want to redefine the indexing routines for those cases where the
default indexing scheme is too inefficient and thereby enhance the efficiency
of the theory manipulation subroutines.

Any symbol in \lisp{} can be a theory, and theories need not be declared.  It is
sufficient to pass a symbol to the indexing or theory manipulation subroutines.

\section{Indexing Subroutines}

\label{index}
The subroutine {\tt index} is used to add sentences to a theory. The subroutine
takes two arguments, the sentence to be added and the theory to which the
sentence is to be added.  Before adding the sentence to the theory, it checks
that no identical sentence is contained in the theory, as determined by {\tt
eq}.  Note that sentences that are {\tt equal} but not {\tt eq} are not
detected by this test; consequently, multiple calls to {\tt index} with
such sentences will result in multiple copies being placed in the theory.  The
subroutine always returns the first argument.

\label{unindex}
The subroutine {\tt unindex} is used to delete sentences from a theory. The
subroutine takes two arguments, the sentence to be deleted and the theory from
which it is to be deleted.  All sentences in the theory that are {\tt eq} to the
first argument are deleted from the theory.  The subroutine always returns the
first argument.

\label{indexps}
The routine {\tt indexps} is used to return all sentences in a theory that
have subexpressions that might unify with the query. The subroutine takes two
arguments, the query expression and the theory in which to search for sentences
with potentially matching subexpressions. However, the list returned may include
sentences that do not unify with the query. The order of the sentences in the
list is the same as the order in whihc the sentences were added to the theory.

The following examples illustrate the behavior of these subroutines.

\medskip
\beginverbatim
User: (index '(brother ralph sue) 'global)
Lisp: (BROTHER RALPH SUE)

User: (index '(sister sue ralph) 'global)
Lisp: (SISTER SUE RALPH)

User: (index '(=> (father ?x ?y) (child ?y ?x)) 'global)
Lisp: (=> (FATHER ?X ?Y) (CHILD ?Y ?X))

User: (index '(sister sue ralph) 'global)
Lisp: (SISTER SUE RALPH)

User: (indexps 'brother 'global)
Lisp: ((BROTHER RALPH SUE))

User: (indexps 'ralph 'global)
Lisp: ((BROTHER RALPH SUE) (SISTER SUE RALPH) (SISTER SUE RALPH))

User: (indexps '(brother ralph sue) 'global)
Lisp: ((BROTHER RALPH SUE))

User: (indexps '(father jim bill) 'global)
Lisp: ((=> (FATHER ?X ?Y) (CHILD ?Y ?X)))

User: (indexps '?x 'global)
Lisp: ((BROTHER RALPH SUE)
       (SISTER SUE RALPH)
       (=> (FATHER ?X ?Y) (CHILD ?Y ?X))
       (SISTER SUE RALPH))

User: (unindex (car (indexps '?x 'global)) 'global)
Lisp: (BROTHER RALPH SUE)

User: (indexps '?x 'global)
Lisp: ((SISTER SUE RALPH)
       (=> (FATHER ?X ?Y) (CHILD ?Y ?X))
       (SISTER SUE RALPH))

User: (unindex '(sister sue ralph) 'global)
Lisp: (SISTER SUE RALPH)

User: (indexps '?x 'global)
Lisp: ((SISTER SUE RALPH)
       (=> (FATHER ?X ?Y) (CHILD ?Y ?X))
       (SISTER SUE RALPH))

User: (unindex (car (indexps '?x 'global)) 'global)
Lisp: (SISTER SUE RALPH)

User: (indexps '?x 'global)
Lisp: ((=> (FATHER ?X ?Y) (CHILD ?Y ?X))
       (SISTER SUE RALPH))
\endverbatim
\medskip

\label{contents}
The subroutine {\tt contents} takes a theory as an argument and returns a list
of sentences stored in the theory. The sentences are ordered from the earliest to
the latest in the returned list.

\medskip
\beginverbatim
User: (index '(mathematician eudoxus) 'global)
Lisp: (MATHEMATICIAN EUDOXUS)

User: (index '(astronomer menaechmus) 'global)
Lisp: (ASTRONOMER MENAECHMUS)

User: (contents 'global)
Lisp: ((MATHEMATICIAN EUDOXUS) (ASTRONOMER MENAECHMUS))
\endverbatim
\medskip

\label{empty}
The subroutine {\tt empty} is used to delete all sentences from a theory.

\medskip
\beginverbatim
User: (index '(mathematician eudoxus) 'global)
Lisp: (MATHEMATICIAN EUDOXUS)

User: (index '(astronomer menaechmus) 'global)
Lisp: (ASTRONOMER MENAECHMUS)

User: (empty 'global)
Lisp: DONE

User: (contents 'global)
Lisp: NIL
\endverbatim

The variable {\tt *theories*} contains a list of theories containing one or more
sentences.  The variable is set automatically by \epilog{}'s indexing routines.

\medskip
\beginverbatim
User: *theories*
Lisp: NIL

User: (index 'sunny 'lax)
Lisp: SUNNY

User: (index 'foggy 'sfo)
Lisp: FOGGY

User: *theories*
Lisp: (SFO LAX)

User: (empty 'LAX)
Lisp: DONE

User: *theories*
Lisp: (SFO)
\endverbatim
\medskip

\label{define-theory}
The subroutine {\tt define-theory} is used to define a theory with many
sentences in one step.  The general form of the subroutine call is {\tt
(define-theory $theory$ $string$ $list$)}. The contents of the theory are
deleted and replaced with the specified sentences using the subroutine {\tt
index}. The $string$ argument is a string that defines the {\tt concept}
documentation for the theory. The subroutine always returns the theory argument.
The documentation for a theory can be examined by using the Lisp subroutine {\tt
documentation}, e.g., {\tt (documentation 'global 'concept)}.

\label{deftheory}
{\tt deftheory} is a macro that produces a call to {\tt define-theory}.  It
takes a theory name, an optional documentation string, and a list of sentences
as arguments.  It empties the specified theory, indexes the specified sentences,
and adds the specified documentation.

\beginverbatim
User: (deftheory global
        "The sentences for the Global theory"
        (father jim ralph)
        (father jim sue)
        (=> (apple ?x) (fruit ?x)))
Lisp: GLOBAL

User: (contents 'global)
Lisp: ((FATHER JIM RALPH)
       (FATHER JIM SUE)
       (=> (APPLE ?X) (FRUIT ?X)))

User: (documentation 'global 'concept)
Lisp: "The sentences for the Global theory"
\endverbatim

\section{Theory Manipulation Subroutines}

The subroutine {\tt save} takes as argument a sentence, a theory, and an
optional equivalence checker.  If the theory contains a sentence that is
equivalent (according to the specified equivalence checker), nothing happens,
and {\tt save} returns {\tt nil}. Otherwise, the specified sentence is added to
the end of the theory.

\medskip
\beginverbatim
User: (save '(parent art bob) 'global)
Lisp: (PARENT ART BOB)

User: (save '(parent art bob) 'global)
Lisp: NIL

User: (contents 'global)
Lisp: ((PARENT ART BOB))

User: (save '(loves ?x ?x) 'global)
Lisp: (LOVES ?X ?X)

User: (save '(loves jill jill) 'global)
Lisp: (LOVES JILL JILL)

User: (save '(loves joe joe) 'global 'unifyp)
Lisp: NIL

User: (save '(loves joe jill) 'global 'unifyp)
Lisp: (LOVES JOE JILL)

User: (contents 'global)
Lisp: ((PARENT ART BOB) (LOVES ?X ?X) (LOVES JILL JILL) (LOVES JOE JILL))
\endverbatim
\medskip

The subroutine {\tt drop} takes as argument a sentence, a theory, and an
optional equivalence checker.  It removes from the specified theory all
sentences equivalent to the specified sentence (according to the specified
equivalence checker).  It returns {\tt done} as value.

\medskip
\beginverbatim
User: (save '(parent art bob) 'global)
Lisp: (PARENT ART BOB)

User: (save '(parent art bea) 'global)
Lisp: (PARENT ART BEA)

User: (save '(parent art bess) 'global)
Lisp: (PARENT ART BESS)

User: (contents 'global)
Lisp: ((PARENT ART BOB) (PARENT ART BEA) (PARENT ART BESS))

User: (drop '(parent art bess) 'global)
Lisp: DONE

User: (contents 'global)
Lisp: ((PARENT ART BOB) (PARENT ART BEA))

User: (drop '(parent ?x ?y) 'global)
Lisp: DONE

User: (contents 'global)
Lisp: ((PARENT ART BOB) (PARENT ART BEA))

User: (drop '(parent ?x ?y) 'global 'unifyp)
Lisp: DONE

User: (contents 'global)
Lisp: NIL
\endverbatim
\medskip

\label{kill}
The subroutine {\tt kill} takes a single expression argument, and is used to
delete all sentences in the current theory that have subexpressions that match
the argument. Independent of the sentences that are deleted, the routine always
returns {\tt done}.  The subroutine {\tt kill} takes a matching subroutine as
optional third argument, which defaults to {\tt samep}. 

\medskip
\beginverbatim
User: (deftheory global
        (father jim ralph)
        (father jim sue)
        (mother jill ralph)
        (=> (apple ?x) (fruit ?x)))
Lisp: GLOBAL

User: (kill 'father 'global)
Lisp: DONE

User: (contents 'global)
Lisp: ((MOTHER JILL RALPH)
       (=> (APPLE ?X) (FRUIT ?X)))

User: (kill '(apple mac) 'global 'unifyp)
Lisp: DONE

User: (contents 'global)
Lisp: ((MOTHER JILL RALPH))
\endverbatim
\medskip

The {\tt truex} command takes an expression, a sentence, and a theory as
arguments.  If the theory contains a matching sentence, {\tt truex} plugs the
variable bindings (if any) into the specified expression and returns the answer. 
There is no restriction on the position of the variables in the query.  The {\tt
truex} subroutine always returns the first answer it finds.

The {\tt truep} subroutine is equivalent to {\tt truex} with {\tt t} as the
first argument.

The {\tt trues} subroutine is similar to {\tt truex} except that it returns a
list of all possible answers.  The order of answers on the list is the order in
which they are found in the theory.

The {\tt trueg} subroutine takes the same arguments as {\tt truex} and {\tt
trues} and returns an answer generator as value.  Each time this generator is
called, it returns a different answer to the orginal question.  When all
answers have been exhausted, it return {\tt nil}. 

\medskip
\beginverbatim
User: (save '(parent art bob) 'global)
Lisp: (PARENT ART BOB)

User: (save '(parent art bea) 'global)
Lisp: (PARENT ART BEA)

User: (save '(parent art bess) 'global)
Lisp: (PARENT ART BESS)

User: (save '(=> (parent ?x ?y) (not parent ?y ?x)) 'global)
Lisp: (=> (PARENT ?X ?Y) (NOT PARENT ?Y ?X))

User: (truep '(parent art bob) 'global)
Lisp: T

User: (truep '(parent ?x bob) 'global)
Lisp: T

User: (truep '(parent @l) 'global)
Lisp: T

User: (truex '?x '(parent ?x bob) 'global)
Lisp: ART

User: (truex '?y '(parent art ?y) 'global)
Lisp: BOB

User: (truex '(related @l) '(parent @l) 'global)
Lisp: (RELATED ART BOB)

User: (trues '?y '(parent art ?y) 'global)
Lisp: (BOB BEA BESS)

User: (trues '(related ?x ?y) '(parent ?x ?y) 'global)
Lisp: ((RELATED ART BOB) (RELATED ART BEA) (RELATED ART BESS))

User: (setq gen (trueg '?y '(parent art ?y) 'global))
Lisp: #<continuation23>

User: (funcall gen)
Lisp: BOB

User: (funcall gen)
Lisp: BEA

User: (funcall gen)
Lisp: BESS

User: (funcall gen)
Lisp: NIL
\endverbatim
\medskip

\section{Composite Theories}

Often in working with theories, it is useful to include the facts from one
theory inside of another theory.  One way to do this is to use the {\tt save}
command to add the sentences to the other theory as well, but this can be
wasteful.  An alternative is to create an {\it inclusion} link between one
theory and another, thereby implicitly including the sentences of the first
theory in the second theory.

The \epilog{} {\it theory heterarchy} is a directed acyclic graph in which the
nodes are theories and the arcs are inclusion links.  The graph must be acyclic
in order to avoid infinite loops in the lookup routines.  {\it Warning: the user
is responsible for ensuring that the graph is cycle-free; \epilog{}'s subroutines
do not test for this condition.}

As an example of the use of the theory mechanism, consider an application with
five theories: a theory of a theory of physics, a theory of economics, a theory
of mathematics, a theory of algebra, and a theory calculus.  We might want to
include our theory of mathematics in our theory of physics and our theory of
economics; and we might want to include our theories of algebra and calculus in
our theory of mathematics.  This can be done by setting up a theory heterarchy
like the one shown below.  The arrows here point from the including theory to
the included theory.  Note that the theory of mathematics includes two other
theories and is in turn included by two other theories.

\vbox to0.97in{}
\centerline{\special{picture heterarchy}\hskip2.00in}
\medskip

In this chapter, we first discuss the subroutines for managing the theory
heterarchy, and then we look at the \epilog{} subroutines for accessing and
modifying {\it composite theories}, i.e. theories with included theories.

\section{Theory Composition Subroutines}

\label{includes}
The subroutine {\tt includes} takes two theories as arguments and modifies the
theory heterarchy so that the first theory includes the second.  {\tt includes}
returns {\tt DONE} as value.

\label{unincludes}
The subroutine {\tt unincludes} takes two theories as arguments and modifies
the theory heterarchy so that the first theory does not include the second.  The
routine always returns {\tt DONE} as value.

\label{decludes}
The subroutine {\tt decludes} is used to remove all includes links for a theory.
The routine returns {\tt DONE} as value.

\label{includees}
The subroutine {\tt includees} takes a theory as argument and returns a list of
theories that it includes. 

\label{includers}
The subroutine {\tt includers} takes a theory as argument and returns a list of
theories in which it is included. 

The following examples illustrate the behavior of these subroutines.

\medskip
\beginverbatim
User: (includes 'c 'global)
Lisp: DONE

User: (includes 'd 'global)
Lisp: DONE

User: (includes 'a 'c)
Lisp: DONE

User: (includes 'a 'd)
Lisp: DONE

User: (includes 'b 'd)
Lisp: DONE

User: (includees 'a)
Lisp: (C D)

User: (includees 'b)
Lisp: (D)

User: (includees 'c)
Lisp: (GLOBAL)

User: (includees 'd)
Lisp: (GLOBAL)

User: (includees 'global)
Lisp: NIL

User: (includers 'a)
Lisp: NIL

User: (includers 'b)
Lisp: NIL

User: (includers 'c)
Lisp: (A)

User: (includers 'd)
Lisp: (A B)

User: (includers 'global)
Lisp: (C D)
\endverbatim
\medskip

\label{*includers*}
The variable {\tt *includers*} is a list of all theories that include one
or more subtheories.  For the graph created earlier, this variable will be
bound to the list {\tt (B A D C)}.

\section{Manipulation Subroutines for Composite Theories}

The theory manipulation subroutines described in the preceding chapter operate
only on individual theories; inclusion links are ignored.  This allows one to
examine and modify individual theories without affecting or being affected by
the theory heterarchy.

In order to allow the user to take advantage of the theory heterarchy, \epilog{}
provides a series of analogous subroutines for accessing composite theories. 
The {\tt knownp} subroutine takes arguments and returns values in a manner
analogous to that of {\tt truep}; {\tt knownx} is analogous to {\tt truex}; {\tt
knowns} is analogous to {\tt trues}; {\tt knowng} is analogous to {\tt trueg}. 
All of these subroutines treat the sentences in included theories as though
stored explicitly in the including theories.

\medskip
\beginverbatim
User: (includes 'top 'left)
Lisp: DONE

User: (includes 'top 'right)
Lisp: DONE

User: (includes 'left 'bottom)
Lisp: DONE

User: (save '(p a) 'top)
Lisp: (P A)

User: (save '(q a) 'left)
Lisp: (Q A)

User: (save '(r a) 'right)
Lisp: (R A)

User: (save '(s a) 'bottom)
Lisp: (S A)

User: (truep '(p a) 'top)
Lisp: T

User: (truep '(q a) 'top)
Lisp: NIL

User: (knownp '(p a) 'top)
Lisp: T

User: (knownp '(q a) 'top)
Lisp: T

User: (knownp '(r a) 'top)
Lisp: T

User: (knownp '(s a) 'top)
Lisp: T
\endverbatim
\medskip

In looking up sentences in a composite theory, these subroutines first check the
sentences stored explicitly in the theory.  They then look at sentences in all
included theories, in the order in which the inclusion links were set up.  The
search is depth-first, so that theories included in an included theory are
examined before moving on to the next included theory.  This behavior is
illustrated in the following example.

\medskip
\beginverbatim
User: (includes 'top 'left)
Lisp: DONE

User: (includes 'top 'right)
Lisp: DONE

User: (includes 'left 'bottom)
Lisp: DONE

User: (save '(p a) 'top)
Lisp: (P A)

User: (save '(p b) 'left)
Lisp: (P B)

User: (save '(p c) 'right)
Lisp: (P C)

User: (save '(p d) 'bottom)
Lisp: (P D)

User: (knowns '?x '(p ?x) 'top)
Lisp: (A B D C)
\endverbatim
\medskip

Note that, if a theory is included in another theory more than once, its
sentences may be examined a second time.  Hence, it is desirable to ensure that
there is no reconvergent fanout in the theory heterarchy.  (It has been
suggested that the subroutines mark included theories so that they are
processed only once.  This would also avoid infinite loops in the case of
cycles in the theory graph.  It is likely that this modification will be made
in future versions of \epilog{}.)

\vfill\eject
%%%%%%%%%%%%%%%%%%%%%%%%%%%%%%%%%%%%%%%%%%%%%%%%%%%%%%%%%%%%%%%%%%%%%%%%%%%%%%%%

\chapter{7}{Utilities}

\section{Viewing Theories}

\label{show}
The subroutine {\tt show} is used to {\it print} information about the sentences
in a theory. The subroutine takes as arguments a pattern (a KIF expression), a
theory, and (optionally) a matching subroutine.  If no third argument is
supplied, {\tt show} uses {\tt matchp}.

The first argument is an expression that specifies the types of sentences in the
theory to be displayed.  All sentences that have subexpressions that match the first
argument are displayed in the order they appear in the theory.  The subroutine
has the side effect of printing the sentences with matching subexpressions and
always returns the value {\tt DONE}.

\medskip
\beginverbatim
User: (deftheory global
        (father jim ralph)
        (father jim sue)
        (=> (apple ?x) (fruit ?x))
        (> 16 3))
Lisp: GLOBAL

User: (show '?x 'global)
Lisp: (FATHER JIM RALPH)
      (FATHER JIM SUE)
      (=> (APPLE ?X) (FRUIT ?X))
      (> 16 3)
      DONE

User: (show 'jim 'global)
Lisp: (FATHER JIM RALPH)
      (FATHER JIM SUE)
      DONE

User: (show 'jill 'global)
Lisp: DONE

User: (show '(father ?a sue) 'global)
Lisp: (FATHER JIM SUE)
      DONE

User: (show '(father ?a sue) 'global 'samep)
Lisp: DONE

User: (show '(fruit ?z) 'global)
Lisp: (=> (APPLE ?X) (FRUIT ?X))
      DONE

User: (show '(fruit fred) 'global)
Lisp: (=> (APPLE ?X) (FRUIT ?X))
      DONE
\endverbatim

\section{Secondary Storage Subroutines}

\label{load-theory}
The subroutine {\tt load-theory} is used to load a set of sentences from a
file into a theory. The sentences in the file can be created using a standard
text editor. The general form of the subroutine call is {\tt (load-theory
$filename$ $theory$ $subroutine$)}.  The first argument specifies the pathname
for the file that contains the sentences; the second is the theory to which the
sentences are to be added; and the third is the subroutine to be used in adding the
sentences to the theory.  The third argument is optional, and its default value is the
subroutine {\tt save}.

The existing contents of the theory are deleted, and the sentences from the file are
added to the theory in the same order as they appear in the file. The subroutine
always returns {\tt DONE}. Since the previous contents of the theory are deleted, it
may be more efficient to use the subroutine {\tt index} to load the sentences from a
file into a theory (instead of the default {\tt save}), but this should be done only
if one is sure the file does not contain duplicate sentences.

\label{load-sentences}
The subroutine {\tt load-sentences} is identical to {\tt load-theory} except
that the existing sentences in the theory are not deleted before loading the
sentences from the file.

\label{dump-theory}
The subroutine {\tt dump-theory} is used to save the entire contents of a theory
to a file. The general form of the subroutine call is {\tt (dump-theory $theory$
$filename$)}. The sentences in the theory {\tt $theory$} are saved in the file
$filename$. The sentences in the file will be in the order that they
appear in the theory.  The subroutine always returns {\tt DONE}.

A useful sequence of operations to edit the contents of a theory is to save
the contents of the theory in a file using the subroutine {\tt dump-theory},
edit the file using a text editor, and finally load the modified file using
the subroutine {\tt load-theory}.

\label{dump-sentences}
The subroutine {\tt dump-sentences} is similar to {\tt dump-theory}, except
that its first argument is a list of sentences versus a theory. The general form
of the subroutine call is {\tt (dump-sentences <list-of-sentences> $filename$)}.
The sentences in the file will be in the same order as they are in the first
argument, and they are not transformed in any way.  The sentences are saved in
the file, and the subroutine returns {\tt DONE}.

\section{Miscellaneous}

The value of the variable {\tt *epilog-version*} is the currently loaded version
of \epilog{}.

The subroutine {\tt reset} empties all theories, breaks all theory inclusion
links, and setsall \epilog{} variables to their initial values.

The subroutine {\tt demo} takes a file name as argument and demonstrates the
contents.  First, it instructs the user to type a carriage return to advance. 
When the carriage return is typed, {\tt demo} reads in one sexpression, prints
it on the terminal, evaluates it, and prints the result on the terminal.  It
then waits for the next carriage return from the user.  {\tt demo} returns {\tt
done} as value.

The subroutine {\tt test} takes a filename as argument and executes each of the
tests in the file.  The file consists of sequence of s-expressions. The odd
numbered s-expressions are the tests to evaluate, and the even numbered
s-expressions are the expected results. An expected result of {\tt *} means that
any value is acceptable. The subroutine returns the total number of errors.

\vfill\eject
%%%%%%%%%%%%%%%%%%%%%%%%%%%%%%%%%%%%%%%%%%%%%%%%%%%%%%%%%%%%%%%%%%%%%%%%%%%%%%%%

\chapter{5}{Inference}

\section{Backward Chaining}

The basic reasoning method used in \epilog{} is an efficient implementation
of the {\it model elimination} proof procedure \cite{Loveland} enhanced to
handle the extensions to first order logic defined in SIF (notably metalevel
information).  There are also some extensions to support procedural attachment
and nonmonotonic reasoning.

The backward chaining subroutines in \epilog{} take arguments and return values
in a manner analogous to the lookup subroutines in \epilog{}.  {\tt findp} is
analogous to {\tt knownp}; {\tt findx} is analogous to {\tt knownx}; {\tt finds}
is analogous to {\tt knowns}; and {\tt findg} is analogous to {\tt knowng}.

The special strength of the inference subroutines is their ability to do
inference with logical information encoded as implications.  Here, we enter a
definition for the {\tt grandparent} relation, and we enter some facts about
Art's family.  Although, according to our definitions, Art is the grandparent of
Cal, {\tt knownp} answers {\tt nil}.  This is the correct answer for {\tt
knownp} -- after all, the fact is not stored explicitly in the theory.  By
contrast, {\tt findp} is able to prove the fact.  The {\tt findx} subroutine is
able to find a grandparent of Cal and a grandchild of Art.  The {\tt finds}
subroutine is able to find all of the grandchildren of Art.  The {\tt findg}
subroutine returns a generator.

\medskip
\beginverbatim
User: (save '(<= (grandparent ?x ?z) (parent ?x ?y) (parent ?y ?z))
            'global)
Lisp: (<= (GRANDPARENT ?X ?Z) (PARENT ?X ?Y) (PARENT ?Y ?Z))

User: (save '(parent art bob) 'global)
Lisp: (PARENT ART BOB)

User: (save '(parent bob cal) 'global)
Lisp: (PARENT BOB CAL)

User: (save '(parent bob coe) 'global)
Lisp: (PARENT BOB COE)

User: (knownp '(grandparent art cal) 'global)
Lisp: NIL

User: (findp '(grandparent art cal) 'global)
Lisp: T

User: (findx '?x '(grandparent ?x cal) 'global)
Lisp: ART

User: (findx '?y '(grandparent art ?y) 'global)
Lisp: CAL

User: (finds '?y '(grandparent art ?y) 'global)
Lisp: (CAL COE)

User: (setq gen (findg '?y '(grandparent art ?y) 'global))
Lisp: #<continuation24>

User: (funcall gen)
Lisp: CAL

User: (funcall gen)
Lisp: COE

User: (funcall gen)
Lisp: NIL
\endverbatim
\medskip

So far, we have concentrated exclusively on relations.  Much, if not most, of
our conceptualization of the world naturally takes the form of functions.  For
functional information, we use a notation much closer to that of \lisp{}.  If a
list is empty, the result of appending the list onto a second list is just
the second list; otherwise, the result is obtained by adding the first element
of the list to the result of appending the rest of the list to the second list.

\medskip
\beginverbatim
User: (save '(= (append (listof) ?m) ?m) 'global)

User: (save '(<= (= (append (listof ?x @l) ?m) (listof ?x @n))
                 (= (append (listof @l) ?m) (listof @n)))
            'global)
\endverbatim
\medskip

Now it is possible to append two lists by making a call to the {\tt findx}
routine.

\medskip 
\beginverbatim
User: (findx '?z '(= (append (listof 1 2) (listof 3 4)) ?z) 'global)
Lisp: (listof 1 2 3 4)
\endverbatim
\medskip

The inference subroutines also include a subroutine, called {\tt findval},
capable of evaluating terms directly.

\medskip 
\beginverbatim
User: (findval '(append (listof 1 2) (listof 3 4)) 'global)
Lisp: (listof 1 2 3 4)
\endverbatim
\medskip

\section{Forward Chaining}

The inference subroutines introduced so far illustrate backward reasoning (from
the goal to premises using backward implications).  \epilog{} is also
capable of forward reasoning (from premises to conclusions using forward
implications).

As an example, consider the following interaction.  By setting {\tt *saves*} to
{\tt (family)}, the user directs the system to save literals involving the {\tt
family} relation.  The forward implication asserts that, if a person is in a
particular family, then all of his children are in that family as well.  By
writing it as a forward implication, we are saying that the implication should be
triggered whenever we get information about a person's family.  The subroutine
{\tt assume} is used to add sentences to the database and conduct all such
forward chaining.  If we discover that Art is a Garfunkel, then we immediately
conclude that Bob and Cal and Coe are Garfunkels as well and these facts are
stored explicitly in the database.

\medskip
\beginverbatim
User: (setq *saves* '(family))
Lisp: (FAMILY)

User: (save '(=> (family ?x ?z) (parent ?x ?y) (family ?y ?z)) 'global)
Lisp: (=> (FAMILY ?X ?Z) (PARENT ?X ?Y) (FAMILY ?Y ?Z))

User: (assume '(family art garfunkel) 'global)
Lisp: DONE

User: (knowns '?x '(family ?x garfunkel) 'global)
Lisp: (ART BOB CAL COE)
\endverbatim
\medskip

\section{Search Control}

All \epilog{} subroutines use iterative deepening search, controlled by the
variables {\tt *start*}, {\tt *increment*}, and {\tt *depth*}.  The value of
{\tt *start*} determines the starting depth cutoff.  On each cycle of iterative
deepening, this cutoff is increased by {\tt *increment*}.  The search
terminates when the cutoff exceeds {\tt *depth*}.  All subroutines set the
variable {\tt terminate*} to {\tt t} if they fail due to a depth cutoff.

\medskip
\beginverbatim
User: (deftheory global
        (p a)
        (p b)
        (q b)
        (<= (r ?x) (p ?x) (q ?x)))
Lisp: GLOBAL

User: (let ((*depth* 0)) (findp '(p a) 'global))
Lisp: NIL

User: *termination*
Lisp: T

User: (let ((*depth* 1)) (findp '(p a) 'global))
Lisp: T

User: (let ((*depth* 1)) (findp '(r b) 'global))
Lisp: NIL

User: *termination*
Lisp: T

User: (let ((*depth* 2)) (findp '(r b) 'global))
Lisp: T

User: (let ((*depth* 1)) (findp '(s b) 'global))
Lisp: NIL

User: *termination*
Lisp: NIL
\endverbatim
\medskip

The value of the variable {\tt *ancestry*} determines whether or not various
inference routines save and check ancestries in processing subgoals.  In the
following examples, we see that, with {\tt *ancestry*} set to {\tt nil}, the
inference is terminated by depth cutoff, whereas, with {\tt *ancestry*} set to
{\tt t}, it is terminated quickly by an ancestry check.

\medskip
\beginverbatim
User: (save '(<= (t ?x) (t ?x)) 'global)
Lisp: (<= (T ?X) (T ?X))

User: (let ((*depth* 10) (*ancestry* nil)) (findp '(t a) 'global))
Lisp: NIL

User: *termination*
Lisp: T

User: (let ((*depth* 10) (*ancestry* t)) (findp '(t a) 'global))
Lisp: NIL

User: *termination*
Lisp: NIL
\endverbatim
\medskip

Under certain circumstances, \epilog{}'s inference routines can be guaranteed
complete even if {\tt *ancestry*} is set to {\tt nil}, e.g. in the case of
theories consisting entirely of Horn clauses.  However, in general,
completeness requires that {\tt *ancestry*} be set to {\tt t}.

The logical constant {\tt cut} can be used within conjunctions, disjunctions,
and rules.  The effect is that same as that of the cut in \prolog{}.  The
following examples illustrate.

\medskip
\beginverbatim
User: (deftheory global
        (p a)
        (p b)
        (q b)
        (<= (r ?x) (p ?x) (q ?x))
        (<= (s ?x) (p ?x) cut (q ?x))
        (<= (s ?x) (q ?x)))
Lisp: GLOBAL

User: (findx '?x '(r ?x) 'global)
Lisp: B

User: (findx '?x '(s ?x) 'global)
Lisp: NIL
\endverbatim
\medskip

\section{Reasoning with Equality}

Equality reasoning requires that the equality axioms be provided explicitly. 
The primary advantage of SIF is that substitutions axioms are not necessary.

As an example of reasoning with equality, consider the following theorem
proving task from teh domain of abstract algebra. Given the axioms asserting the
existence of left and right identities and a right inverse for the {\tt *}
operator, we ask \epilog{} to prove that the right inverse is also a left
inverse.

\medskip 
\beginverbatim
User: (deftheory group
        (<= (= (* ?x ?y) ?x) (= ?y e))
        (<= (= (* ?y ?x) ?x) (= ?y e))
        (<= (= (* ?x ?y) e) (= ?y (inv ?x)))
        (<= (= (* ?x ?v) ?w)
            (= (* ?y ?z) ?v)
            (= (* ?x ?y) ?u)
            (= (* ?u ?z) ?w)))
Lisp: GROUP

User: (deftheory equality
        (= ?x ?x)
        (<= (= ?x ?y) (= ?y ?x))
        (<= (= ?x ?z) (= ?x ?y) (= ?y ?z)))
Lisp: EQUALITY

User: (includes 'group 'equality)
Lisp: DONE

User: (setq *depth* 5)
Lisp: 5

User: (findp '(= (* (inv x) x) e) 'group)
Lisp: T
\endverbatim
\medskip

\section{Procedural Attachments}

\epilog{} provides procedural attachments on all of KIF's arithmetic relations. 
These procedural attachments allow the inference routines to succeed whenever
working on ground arithmetic literals.  They do {\it not} necessarily succeed on
non-ground literals, though it is relatively easy to implement generators for all
solutions to such problems.

\medskip 
\beginverbatim
User: (findp '(> 3 2) 'global)
Lisp: t

User: (findp '(not (> 2 3)) 'global)
Lisp: t

User: (findx '?x '(> 3 ?x) 'global)
Lisp: nil
\endverbatim
\medskip

Predefined terms are evaluated using the {\tt ==} relation.  In this case, the
second argument to {\tt ==} may be a variable.

\medskip 
\beginverbatim
User: (findp '(== (+ 2 2) 4) 'global)
Lisp: t

User: (findx '?x '(== (+ 2 2) ?x) 'global)
Lisp: 4
\endverbatim
\medskip

Such terms are {\it not} evaluated when used in {\tt =} sentences, unless the
database contains an appropriate axiom linking {\tt =} to {\tt ==}.

\medskip 
\beginverbatim
User: (findp '(= (+ 2 2) 4) 'global)
Lisp: NIL

User: (findx '?x '(= (+ 2 2) ?x) 'global)
Lisp: NIL

User: (save '(<= (= ?x ?y) (== ?x ?y)) 'global)
Lisp: (<= (= ?X ?Y) (== ?X ?Y))

User: (findp '(= (+ 2 2) 4) 'global)
Lisp: T

User: (findx '?x '(== (+ 2 2) ?x) 'global)
Lisp: 4
\endverbatim
\medskip

It is possible to ``implement'' arbitrary procedural attachments using the {\tt
execute} relation.  Whenever \epilog's inference procedure encounters an atomic
sentence involving {\tt execute}, as either a goal or a conclusion, it uses
\lisp{}'s {\tt eval} routine to evaluate the argument.  If the result is
non-null and the sentence is a goal, the goal succeeds.

\medskip 
\beginverbatim
User: (findp '(execute (listp '(1 2 3))) 'global)
Lisp: t

User: (findval '(execute (length '(1 2 3))) 'global)
Lisp: 3

User: (assume '(execute (princ "Hello!")) 'global)
Lisp: Hello!
Lisp: t
\endverbatim
\medskip

\section{Tracing}

This section describes the tracing subroutines in \epilog{}. These subroutines
are useful in acquiring justifications for conclusions and in debugging
unexpected behavior (by identifying incorrect facts).  These subroutines cannot
identify missing facts in the knowledge base directly, though it may be possible
to infer a missing fact from a trace.

Whenever an \epilog{} subroutine attempts to prove a literal that matches a
traced expression, the tracing routines print {\tt Call:} followed by the
literal.  An {\tt Exit:} is followed by an instance of the literal that is
proved.  The inference routines try other choices if the proof attempt for
one choice fails. This is indicated by a {\tt Redo:} which is followed by the
literal for which a new proof is attempted. If a proof for a sentence cannot be
found, the tracing routines print {\tt Fail:} followed by the failing literal. 

\label{trace-expression}
The subroutine {\tt trace-expression} takes any number of expressions as
arguments. It sets up data structures so that \epilog{} subroutines print out
appropriate messages whenever they attempt to prove literals that are {\it
instances} of the specified expressions.  Goal literals are matched with the
traced expressions using the {\tt matchp} subroutine of \epilog{}. It is possible
to trace several patterns in a single call to {\tt trace-expressions} or
in separate calls.  If {\tt trace-expressions} is called without arguments, the
result is a list of the currently traced expressions.

\label{untrace-expression}
The subroutine {\tt untrace-expression} takes any number of expressions as
arguments.  It deletes the specified expressions from the data structures set
up by {\tt trace-expression} and thus turns off the corresponding tracing. All
traced expressions are deleted if no arguments are passed, and a list of the
untraced expressions is returned.  It is possible to untrace several expressions
in a single call to {\tt untrace-expression} or in separate calls.

\vfill\eject
%%%%%%%%%%%%%%%%%%%%%%%%%%%%%%%%%%%%%%%%%%%%%%%%%%%%%%%%%%%%%%%%%%%%%%%%%%%%%%%%

\chapter{n}{Converting KIF to SIF}

Simplified Interchange Format (SIF) is a proper subset of Knowledge Interchange
Format (KIF).  Every expression in SIF is an expression in KIF, but not every
expression in KIF is an expression in SIF.

By and large, SIF is much simpler than KIF.  (Hence the name.)  There are no
definitions and no nonmonotonic rules; there are no explicit quantifiers; there
are no embedded implications; and there are no complicated operators (like {\tt
setofall}, {\tt lambda}, and so forth).  What's more, nested terms are
prohibited (except for designators, as described in the last chapter).

These restrictions dramatically simplify the task of automated reasoning.  In
particular, they allow us to implement an inference procedure that is both
efficient and easily understandable.

Despite the subset relationship between SIF and KIF, SIF is every bit as
expressive as KIF, i.e. for any set of KIF sentences, there is an equivalent set
of SIF sentences.  \epilog{} provides subroutines capable of transforming KIF
sentences into equivaent SIF sentences.

The {\tt sif} subroutine converts arbitrary KIF sentences into SIF sentences in
boolean form (i.e. without occurrences of {\tt <=} or {\tt =>}).  It skolemizes,
converts non-primitive nested terms into equational conditions, and then
converts to boolean form.  This is especially usefully for converting negated
goals to SIF.  The following example assumes that {\tt b} is a name
and that {\tt f} and {\tt g} are pseudofunctionals.

\medskip
\beginverbatim
User: (sif '(not (r (f (g ?x)) b)))
Lisp: (OR (NOT (= (G ?X) ?X1)) (NOT (= (F ?X1) ?X2)) (NOT (R ?X2 B)))
\endverbatim
\medskip

The {\tt rules} subroutine converts arbitrary KIF sentences into SIF rules.  It
skolemizes, converts non-primitive nested terms into equational conditions, and
then converts to backward rule form.  In the following examples, we again assume
that {\tt b} is a name and that {\tt f} and {\tt g} are pseudofunctionals.

\medskip
\beginverbatim
User: (car (rules '(r (f (g ?x)) b)))
Lisp: (<= (R ?X2 B) (= (F ?X1) ?X2) (= (G ?X) ?X1))

User: (car (rules '(<= (r (f ?x) ?y) (p ?x) (q (g ?y))))
Lisp: (<= (R ?X1 ?Y) (= (F ?X) ?X1) (P ?X) (= (G ?Y) ?Y1) (Q ?Y1))
\endverbatim
\medskip

The {\tt rules} subroutine is guided by the settings of the variables {\tt
*names*} (default {\tt t}) and {\tt *functionals*} (default {\tt nil}).  If
{\tt *names*} is {\tt t}, all object constants are assumed to be {\it
primitive}.  If it is a list, only those object constants on the list are
assumed to be primitive.  If {\tt *functionals*} is {\tt t}, all function constants
are assumed to produce primitive terms when applied to primitive arguments.  If
it is a list, only those function constants on the list are assumed to produce
primitive terms.  {\tt sif} does not modify terms that, accroding to the settings
of these variables are determined to be primitive.

\medskip
\beginverbatim
User: (setq *functionals* '(f))
Lisp: (F)

User: (car (rules '(r (f (g ?x)) b)))
Lisp: (<= (R (F ?X1) B) (= (G ?X) ?X1))

User: (setq *functionals* '(g))
Lisp: (G)

User: (car (rules '(r (f (g ?x)) b)))
Lisp: (<= (R ?X1 B) (= (F (G ?X)) ?X1))

User: (setq *functionals* '(f g))
Lisp: (F G)

User: (car (rules '(r (f (g ?x)) b)))
Lisp: (R (F (G ?X)) B)

User: (setq *names* nil)
Lisp: NIL

User: (car (rules '(r (f (g a)) b)))
Lisp: (<= (R (F (G ?X)) ?Y) (= B ?Y) (= A ?X))
\endverbatim
\medskip

\vfill\eject
%%%%%%%%%%%%%%%%%%%%%%%%%%%%%%%%%%%%%%%%%%%%%%%%%%%%%%%%%%%%%%%%%%%%%%%%%%%%%%%%

\nochapter{Bibliography}

\def\bib#1{\bigskip\noindent#1\par}

\bib{Epistemics: {\it EPIC 1.0 for LISP}, Epistemics Inc., 1994.}

\bib{Genesereth, M. R., Fikes, R. E. et al.  Knowledge Interchange Format
Version 3 Reference Manual, Logic-92-1, Stanford University Logic Group, 1992.}

\bib{Loveland, D.}

\vfill\eject
%%%%%%%%%%%%%%%%%%%%%%%%%%%%%%%%%%%%%%%%%%%%%%%%%%%%%%%%%%%%%%%%%%%%%%%%%%%%%%%%

\nochapter{EPILOG Variables}

\def\variable#1#2{\bigskip\noindent\hangindent=\parindent{\tt #1}\hfil\break#2}

\variable{*ancestry*}{The value of the variable {\tt *ancestry*} determines
whether or not various inference routines save and check ancestries in
processing subgoals.  The initial value is {\tt nil}.}

\variable{*depth*}{The variable {\tt *depth*} has as value a positive integer
indicating the depth of search for iterative deepening.  The initial value is
1000000.}

\variable{*epilog-version*}{The value of {\tt *epilog-version*} is the version
of \epilog{} currently loaded.}

\variable{*functionals*}{{\tt *functionals*} is a variable that determines which
function constants are taken as primitive.  If the value of {\tt *functionals*} is
a list, the elements of the list are considered primitive.  If the value is a
non-list, all atoms are treated as primitives.  The initial value is {\tt nil}.}

\variable{*increment*}{The value of {\tt *increment*} is the amount by which the
depth cutoff is incremented on each round of iterative deepening.  The initial
value is 1000.}

\variable{*inferences*}{The variable {\tt *inferences*} records the number of
inference steps in the current inference process.  The value is set
automatically after each inference.}

\variable{*names*}{{\tt *names*} is a variable that determines which object
constants are taken as primitive.  Characters, strings, and numbers are always
primitive.  If the value of {\tt *names*} is a list, the elements of the list
are considered primitive as well.  If the value is a non-list, all atoms are
treated as primitives.  The initial value is {\tt t}.}

\variable{*saves*}{The value of the variable {\tt *saves*} determines which
literals are saved in forward chaining.  If the value of {\tt *saves*} is a
list, then a derived literal is saved if and only if the first logical constant,
function constant, or relation constant is an item on this list.  If the value
is anything other than a list, all derived literals are saved.  The initial
value is {\tt nil}, i.e. nothing is saved.}

\variable{*start*}{The value of {\tt *start*} is the initial depth cutoff for
iterative deepening.  The initial value is 1000.}

\variable{*termination*}{The variable {\tt *termination*} records whether the
most recent depth-limited search attempt ended because of a depth cutoff.  This
value is set automatically after each inference.}

\variable{*trace-device*}{The value of the variable {\tt *trace-device*} is the
device to which inference trace information is printed.  The default is {\tt t},
which directs all inference routines to print traces on the terminal.}

\vfill\eject
%%%%%%%%%%%%%%%%%%%%%%%%%%%%%%%%%%%%%%%%%%%%%%%%%%%%%%%%%%%%%%%%%%%%%%%%%%%%%%%%

\nochapter{EPILOG Subroutines}

\def\routine#1#2{\bigskip\noindent\hangindent=\parindent{\tt #1}\hfil\break#2}

\routine{(assume $sentence$ $theory$>)}{The subroutine {\tt assume} takes a
sentence and a theory as arguments.  The sentence is assumed to be a literal. 
{\tt assume} uses model elimination to derive conclusions from the specified
sentence and the sentences in the specified theory and its included theories. 
The search is bottom-up, depth-first, and statically-ordered and uses {\tt
*depth*} as a depth limit.  If it derives a literal with in which the principal
conatsnt if on the list {\tt *saves*}, it saves the literal it derives into the
specified theory. {\tt assume} returns {\tt done} as value.}

\routine{(discard $expression$ $theory$)}{The subroutine {\tt discard} takes an
expression and a theory as arguments and forgets all sentences containing an
instance of the specified expression from the specified theory.}

\routine{(findg $term$ $sentence$ $theory$)}{The subroutine {\tt findg} takes as
argument a term, a sentence, and a theory.  It returns a continuation that on
each call tries to prove the specified sentence from the specified theory and
its included theories.  The search is done in iterative deepening fashion,
controlled by the variables {\tt *start*}, {\tt *increment*}, and {\tt *depth*}. 
If the continuation is able to prove the sentence, it returns a copy of the
specified term with variables replaced by values obtained during the proof
process.  After each successful attempt, the continuation can be called again to
get the next answer.  Once all answers have been enumerated, the continuation
returns {\tt nil}.}

\routine{(findp $sentence$ $theory$)}{The subroutine {\tt findp} takes a sentence
and a theory as arguments.  It tries to prove the sentence from the specified
theory and its included theories using model elimination.  The search is done in
iterative deepening fashion, controlled by the variables {\tt *start*}, {\tt
*increment*}, and {\tt *depth*}.  If {\tt findp} is able to prove the sentence,
it returns {\tt t}; otherwise, it returns {\tt nil}.}

\routine{(finds $term$ $sentence$ $theory$)}{The subroutine {\tt finds} takes as
argument a term, a sentence, and a theory.  It tries to prove the specified
sentence from the specified theory and its included theories using model
elimination.  The search is done in iterative deepening fashion, controlled by
the variables {\tt *start*}, {\tt *increment*}, and {\tt *depth*}.  If {\tt
finds} succeeds in proving the sentence, it returns a list of copies of the
specified term, one for each way the sentence can be proved.  In each copy, the
variables are replaced by values obtained during the proof process.  If the
sentence cannot be proved, {\tt finds} returns {\tt nil}.}

\routine{(findx $term$ $sentence$ $theory$)}{The subroutine {\tt findx} takes as
argument a term, a sentence, and a theory.  It tries to prove the specified
sentence from the specified theory and its included theories using the model
elimination.  The search is done in iterative deepening  fashion, controlled by
the variables {\tt *start*}, {\tt *increment*}, and {\tt *depth*}.  If {\tt
findx} is able to prove the sentence, it returns a copy of the specified term
with variables replaced by values obtained during the proof process.  If it
fails to prove the sentence, the value is {\tt nil}.}

\routine{(findval $term$ $theory$)}{The {\tt findval} subroutine takes a term and
a theory as argument.  It uses {\tt findg} to find a term that is provably equal
to the specified term given the sentences in the specified theory.  If it
succeeds, it returns the primitive term as value; otherwise, it returns {\tt
nil}.}

\routine{(forget $sentence$ $theory$)}{The subroutine {\tt forget} takes a
sentence and a theory as arguments.  {\tt forget} uses model elimination to
derive conclusions from the specified sentence and the sentences in the
specified theory and its included theories.  The search is bottom-up,
depth-first, and statically-ordered and uses {\tt *depth*} as a depth limit. 
If it derives a literal with in which the principal constant is on the list
{\tt *saves*}, it drops the literal from the specified theory.  {\tt forget}
returns {\tt done} as value.}

\routine{(primitivep $term$)}{The subroutine {\tt primitivep} takes a term as
argument.  It returns {\tt t} if the term is primitive; otherwise, it returns
{\tt nil}.}

\routine{(pseudoprimitivep $term$)}{The subroutine {\tt pseudoprimitivep}
takes a term as argument.  It returns {\tt t} if the term is pseudoprimitive;
otherwise, it returns {\tt nil}.}

\routine{(pseudosentencep $sentence$)}{The subroutine {\tt pseudosentencep}
takes a sentence as argument.  It returns {\tt t} if the sentence is 
pseudoprimitive; otherwise, it returns {\tt nil}.}

\routine{(rules $sentence$)}{The subroutine {\tt rules} takes a sentence as
argument and returns a list of equivalent backward rules in simplified
interchange format.}

\routine{(sif $sentence$)}{The subroutine {\tt sif} takes a sentence as argument,
converts to a boolean sentence in simplified interchange format, and returns the
result.}

\routine{(sifp $sentence$)}{The subroutine {\tt sifp} takes a sentence as
argument.  It returns {\tt t} if the sentence is in simplified interchange
format; otherwise, it returns {\tt nil}.}

\routine{(trace-expression $expression_1$ ... $expression_n$)}{The subroutine
{\tt trace-expression} takes any number of expressions as arguments.  It sets up 
data structures so that various proof procedures print out appropriate messages
whenever they examine expressions that are instances of the specified
expressions.  If no arguments are passed to {\tt trace-expression}, the result
is a list of currently traced expressions.  Otherwise, the value is {\tt done}.}

\routine{(untrace-expression $expression_1$ ... $expression_n$)}{The subroutine
{\tt untrace-expression} takes any number of expressions as arguments.  It
eliminates the specified expressions from the data structures set up by
{\tt trace-expression} and thus turns off the corresponding tracing.  If no
arguments are passed to{\tt untrace-expression}, all traced expressions are
deleted, and a list of the those expressions is returned as value.  Otherwise,
the value is {\tt done}.}

\vfill\eject
%%%%%%%%%%%%%%%%%%%%%%%%%%%%%%%%%%%%%%%%%%%%%%%%%%%%%%%%%%%%%%%%%%%%%%%%%%%%%%%%

\nochapter{General Index}

\def\index#1#2{\noindent #1\leaders\hbox to 1em{\hss.\hss}\hfill#2\par}

\index{\tt *ancestry*}{n}
\index{\tt *depth*}{n}
\index{\tt *epilog-version*}{n}
\index{\tt *functionals*}{n}
\index{\tt *increment*}{n}
\index{\tt *inferences*}{n}
\index{\tt *names*}{n}
\index{\tt *saves*}{n}
\index{\tt *start*}{n}
\index{\tt *termination*}{n}
\index{\tt *trace-device*}{n}
\index{\tt assume}{n}
\index{\tt discard}{n}
\index{\tt findg}{n}
\index{\tt findp}{n}
\index{\tt finds}{n}
\index{\tt findval}{n}
\index{\tt findx}{n}
\index{\tt forget}{n}
\index{Iterative deepening}{n}
\index{Model elimination}{n}
\index{Occur check}{n}
\index{\tt primitivep}{n}
\index{\tt pseudoprimitivep}{n}
\index{\tt pseudosentencep}{n}
\index{\tt rules}{n}
\index{\tt sif}{n}
\index{\tt trace-expression}{n}
\index{\tt untrace-expression}{n}

\bye

%%%%%%%%%%%%%%%%%%%%%%%%%%%%%%%%%%%%%%%%%%%%%%%%%%%%%%%%%%%%%%%%%%%%%%%%%%%%%%%%
%%%%%%%%%%%%%%%%%%%%%%%%%%%%%%%%%%%%%%%%%%%%%%%%%%%%%%%%%%%%%%%%%%%%%%%%%%%%%%%%
%%%%%%%%%%%%%%%%%%%%%%%%%%%%%%%%%%%%%%%%%%%%%%%%%%%%%%%%%%%%%%%%%%%%%%%%%%%%%%%%
